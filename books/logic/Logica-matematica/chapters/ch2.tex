\chapter{Lógica proposicional}

\begin{theorem} Suponha que uma propriedade vale para toda fórmula atômica e que, se vale para as fórmulas A e B, também vale para $( \neg A), (A \land B), (A \lor B), (A \rightarrow B) e (A \leftrightarrow B)$. Então essa propriedade vale para todas fórmulas da linguagem da lógica proposicional.
\end{theorem}

\begin{solucao}
  null
\end{solucao}

\begin{theorem} Para toda fórmula A, uma, e apenas uma, das afirmações abaixo é verdadeira:
  
  \begin{enumerate}[label=(\alph{*})]
		\item A é uma fórmula atômica
		\item $\exists ! B \mid  A = (\neg B) $.
		\item $\exists B, C \mid A = (B \land C)$.
	\end{enumerate}
  
\end{theorem}

\begin{solucao}
  null
\end{solucao}

\setcounter{th}{10}
\begin{theorem} Para todas fórmulas A e B valem :

  \begin{enumerate}[label=(\alph{*})]
		\item A é uma tautologia se, e somente se, $\neg A$ é uma contradição
		\item A é uma contradição se, e somente se, $\neg A$ é uma tautologia.
		\item A e B são equivalentes se, e somente se, $A \leftrightarrow B $ é uma tautologia.
		\item Se A é uma tautologia e p é uma fórmula atômica, então, se substituir-
    mos todas as ocorrências de p, em A, pela fórmula B, a fórmula obtida
    será uma tautologia;
		\item Se A e $A \leftrightarrow B$ são tautologias então B é uma tautologia.

	\end{enumerate}
  
\end{theorem}

\begin{solucao}
  null
\end{solucao}

