\documentclass[portuguese,12pt,a4paper]{book}
\usepackage[utf8]{inputenc}
\usepackage[T1]{fontenc}
\usepackage{graphicx}
\usepackage{mathtools}
\usepackage{amssymb}
\usepackage{amsthm}
\usepackage{babel}
\usepackage[colorlinks=true]{hyperref}
\usepackage{enumitem}
\usepackage[
	backend=bibtex,
	style=alphabetic,
	sorting=ynt
]{biblatex}
\addbibresource{bibliografia.bib}

\title{A teoria dos conjuntos e os fundamentos da matemática - Soluções}

\newcommand{\cqd}{\hfill $\square$}
\newenvironment{solucao}[1][]{\noindent\textbf{Solução#1:} }{\cqd}
\newcounter{ex}
\newtheorem{exercicio}[ex]{Exercício}
\DeclareMathOperator{\dom}{dom}

\begin{document}
	
	\maketitle
	\tableofcontents
	
	chapter{}

	\chapter{A linguagem da Teoria dos Conjuntos}

\setcounter{ex}{0}

\begin{exercicio}
	Usando a linguagem de primeira ordem da teoria de conjuntos, escreva fórmulas para representar as seguintes frases.
	\begin{enumerate}[label=(\alph{*})]
		\item Não existe o conjunto de todos os conjuntos.
		\item Existe um único conjunto vazio.
		\item x é um conjunto unitário
		\item Existe um conjunto que tem como elemento apenas o conjunto vazio
		\item y é o conjunto dos subconjuntos de x 
	\end{enumerate}
\end{exercicio}

\begin{solucao}
	\begin{enumerate}[label=(\alph{*})]
		\item $\neg \exists x \forall y (y \in x)$ ou $\forall x \exists y (y \notin x)$.
		\item $\exists ! x \forall y (y \notin x)$.
		\item $\exists ! y (y \in x)$.
		\item $\exists x \forall y ((y \in x) \leftrightarrow y = \phi)$.
		\item $\forall z (z \in y \leftrightarrow \forall w (w \in z \to w \in x))$.
	\end{enumerate}
\end{solucao}

\begin{exercicio}
	Marque as ocorrências de variáveis livres nas fórmulas abaixo
\end{exercicio}
\begin{enumerate}[label=(\alph{*})]
	\item $(\forall x (x=y)) \rightarrow (x \in y ) $
	\item $ \forall x ((x=y) \rightarrow (x \in y))$
	\item $\forall x(x=x) \rightarrow (\forall y \exists Z ((x=y) \land (y=z)) \rightarrow \neg(x\in y))$
	\item $ \forall x \exists y(\neg(x=y) \land \forall z ((x \in y) \leftrightarrow \forall w ((w \in z ) \rightarrow (w \in x )))) $
	\item $(x=y)\rightarrow \exists (x=y) $
\end{enumerate}

\begin{solucao}
	\begin{enumerate}[label=(\alph{*})]
    \item $x$ e $y$
    \item $y$ 
    \item $x$
    \item Não há variáveis livres.
    \item $x$ e $y$
\end{enumerate}
\end{solucao}

\begin{exercicio}
	Escreva as subfórmulas de cada fórmula do exercício 2.
\end{exercicio}

\begin{solucao}

\begin{enumerate}[label=(\alph{*})]
	\item 
	\begin{itemize}
		\item $(\forall x (x = y)) \rightarrow (x \in y)$
		\item $(\forall x (x = y))$
		\item $(x = y)$
		\item $(x \in y)$
	\end{itemize}
	
	\item 
	\begin{itemize}
		\item $\forall x ((x = y) \rightarrow (x \in y))$
		\item $(x = y) \rightarrow (x \in y)$
		\item $(x = y)$
		\item $(x \in y)$
	\end{itemize}
	
	\item 
	\begin{itemize}
		\item $\forall x (x = x) \rightarrow (\forall y \exists z (((x = y) \land (y = z)) \rightarrow \neg (x \in y)))$
		\item $\forall x (x = x)$
		\item $(x = x)$
		\item $\forall z \exists y (((x = y) \land (y = z)) \rightarrow \neg (x \in y))$
		\item $((x = y) \land (y = z))$
		\item $(x = y)$
		\item $(y = z)$
		\item $\neg (x \in y)$
		\item $(x \in y)$
	\end{itemize}
	
	\item 
	\begin{itemize}
		\item $(x = y) \rightarrow \exists y (x = y)$
		\item $(x = y)$
		\item $\exists y (x = y)$
	\end{itemize}
\end{enumerate}

\end{solucao}
	chapter{}

	\chapter{Produto Cartesiano, Relações e Funções}
	\chapter{Axioma da Escolha e suas Aplicações}
	chapter{}

	chapter{}

	chapter{}

	chapter{}

	\chapter{Noções de Teoria dos Modelos}
	\chapter{Modelos para ZFC}
	\chapter{\emph{Forcing}}
	
	\printbibliography[title=Referências]
	
\end{document}